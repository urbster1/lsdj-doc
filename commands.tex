\chapter{Commands}

Commands can be used in phrases and tables for altering the sound. There is a lot of power hidden in the commands, so it is suggested that you skim through this chapter at least once to get an idea of what they can do for you.

\includegraphics[width=1cm]{tip}TIP!
\begin{itemize}
        \item \textit{Tapping \textsc{a} on a command letter will display a scrolling help text in the top of the screen. \textsc{a+cursor} can then be used to browse through the existing commands. The text can be paused by holding \textsc{select}.}
	%\marginpar{\includegraphics[width=1cm]{tip}TIP!}
	\end{itemize}

\section{A: Table Start/Stop}

Starts or stops tables in the current channel. Use the table number you want to start, or 20 for stopping.

\begin{description}
\item[A03] start table 3
\item[A20] stop table
\end{description}

\section{B: mayBe}

B determines the probability that notes will play. It can also be used to probabilistically hop in a table.

\subsection{B in Phrases}

Controls how likely it is that the note or sample(s) to the left will be played. The first digit sets probability for the left kit, the second digit sets the probability for notes and right kit.

\begin{description}
    \item[B00] Never play note
    \item[B0F] Always play note/right kit sample
    \item[BF0] Always play left kit sample
	\item[B08] Play note/right kit sample about 50% of the time
\end{description}

\subsection{B in Tables}

In the table screen, B is used to make a hop that only happens sometimes. The first digit sets probability, the second digit sets destination row.

\begin{description}
\item Example:
\item[BF5] Hop to row 5, 15 times out of 16.
\item[B84] Hop to row 4, about 50% of the time.
\item[B03] Never hop to row 3
\end{description}

\section{C: Chord}

\subsection{For Pulse and Wave Instruments:}

\label{command-chord}
Produce chords by doing a simple arpeggio that extends the root note with the given semitones.

\begin{description}
\item[C37] plays a minor chord: 0, 3, 7, 0, 3, 7, 0, 3, 7, \ldots
\item[C47] plays a major chord: 0, 4, 7, 0, 4, 7, 0, 4, 7, \ldots
\item[C0C] plays 0, 0, C, 0, 0, C, 0, 0, C, \ldots
\item[CC0] plays 0, C, 0, C, 0, C, \ldots
\item[CCC] plays 0, C, C, 0, C, C, 0, C, C, \ldots
\item[C00] resets chord
\end{description}

\subsection{For Noise Instruments:}

Applies S command with the given value every second tick.

\section{D: Delay}

Delay the triggering of a note by the given number of ticks.

\section{E: Amplitude Envelope}

This command functions in two different ways, depending on which instrument type it is used on.

\subsection{For Pulse and Noise Instruments}
The first digit value sets the initial amplitude (0=min, \$F=max); the second digit sets the release (0,8: no change, 1-7: decrease, 9-\$F: increase).

\subsection{For Wave Instruments}
\begin{description}
\item[E00] volume 0\%
\item[E01] volume 25\%
\item[E02] volume 50\%
\item[E03] volume 100\%
\end{description}

\section{F: Wave Frame/Finetune}

\subsection{For Pulse Instruments:}
The first digit sets \textsc{pu2 tsp}, the second \textsc{finetune}.
See section~\ref{detune}.

\subsection{For Kit Instruments:}
Modifies the sample position. \$00-\$7F steps forward, \$80-\$FF steps back.

\subsection{For Wave Instruments:}
Change the wave frame that's being played on the wave channel. This command is relative, meaning that the command value will be added to the current frame number. This can be used for playing synth sounds manually.

\includegraphics[width=1cm]{tip}TIP!
\begin{itemize}
        \item \textit{Since a synth sound contains 16 (\$10) waves, issuing the command \textsc{F10} will in effect jump to the next synth sound.}
	%\marginpar{\includegraphics[width=1cm]{tip}TIP!}
	\end{itemize}

\begin{description}
\item Example:
\item[F01] If wave frame 3 is being played, advance 1 frame and start playing frame number 4.
\end{description}

\section{G: Groove Select}

Select the groove to use when playing phrases or tables.

\begin{description}
\item Example:
\item[G04] select groove 4
\end{description}

\section{H: Hop}

H hops to a new play position. It can also be used to stop playing.

\subsection{H in Phrases}

\begin{description}
    \item[H00-H0F] Hop to next phrase. The digit sets destination phrase step.
    \item[H10-HFE] Hop back within the phrase. The first digit sets number of times to hop back, the second digit sets destination step.
    \item[HFF] Stop playing song (or channel, if in live mode).
\end{description}

\includegraphics[width=1cm]{tip}TIP!
\nolinebreak
\begin{itemize}
        \item \textit{If you want to compose in waltz time (3/4), put \textsc{H00} commands on step \textsc{C} in every phrase.}
\end{itemize}

\subsection{H in Tables}

In the table screen, H is used for creating table loops. The first digit sets how many times the hop should be done before moving on; 0 means ``forever.'' The second digit sets the table step to jump to. Loops can be nested; that is, you can have smaller loops inside bigger ones.

\begin{description}
\item Example:
\item[H21] Hop twice to table position 1.
\item[H04] Hop to table position 4 forever.
\end{description}

\section{K: Kill Note}

\begin{description}
\item Example:
\item[K00] Kill note instantly
\item[K03] Kill note after 3 ticks
\end{description}

\section{L: Slide}

Slides to the target note in the given duration. If the instrument's \textsc{pitch} setting is \textsc{tick}, the duration is given in ticks, otherwise in n/360 seconds.

Example:

\begin{verbatim}
  C-4 ---
  F-4 L40
  --- ---
  C-4 L10
\end{verbatim}

This will result in a slide that starts with C-4, bends to F-4, and then quickly bends back to C-4.

\subsection{L in Tables}

The L command also works in the left table command column, using the transpose column to set target note relative to the root note.

\begin{figure}[htbp]
	\begin{center}
		\fbox{\includegraphics{table-slide}}
	\end{center}
	% \caption{Synth Screen}
	% \label{fig:synth}
\end{figure}

Regular transposes and slides are added together independently. In the above example, step 0 transposes one octave up. In step 1, the L command starts sliding one octave down while keeping the transpose from step 0 unchanged. In step 2, the L command is allowed to play out while the transpose stays one octave up. After some time, L will stop one octave down, cancelling out the transpose and in practice returning to the root note.

\section{M: Master Volume}

This command changes the master output volume. The first digit modifies the left output, the second digit the right. The volume can either be set with an absolute value, or changed by a relative value.

Values 0-7 are used to specify absolute volumes. Values 8-\$F give the volume a relative change; 8 is no change, 9-\$B increase, \$D-\$F decrease.

\begin{description}
\item Examples:
\item[M77] Maximize volume
\item[M08] Minimize left volume, leave right volume unchanged
\item[M99] Increase volume with 1 step
\item[MFE] Decrease left volume with 1 step, right volume with 2 steps
\end{description}

\section{O: Set Output}

Pan channel to left, right, none or both outputs.

\section{P: Pitch Bend}

\subsection{For Pulse, Wave and Kit Instruments:}

Does a pitch change with the given speed. The behavior depends on the instrument's \textsc{pitch} setting:

\begin{description}
    \item[\textsc{drum}] Logarithmic pitch bend that updates at 360 Hz. Useful for drum sounds.
	\item[\textsc{fast}] Pitch bend that updates at 360 Hz.
    \item[\textsc{tick}] Pitch bend that updates every tick. Speed can be controlled by changing \textsc{fx/speed}.
    \item[\textsc{step}] Immediate pitch change without bend.
\end{description}

Example:

\begin{description}
\item[P02] Pitch change up with speed 2.
\item[PFE] Pitch change down with speed 2. (\$FE=-2)
\end{description}

\subsection{For Noise Instruments:}

Applies S command with the given value every tick.

\section{R: Retrig the Latest Played Note}

Play the latest played note again. The first digit modulates the volume (0=no change, 1-7=increase, 8-\$F=decrease). The second digit sets a period for the retriggering, zero being the fastest and \$E the slowest. Second digit of \$F means only retrig once.

\begin{description}
\item Example:
\item[R00] very fast retriggering
\item[R0F] retrigger once
\item[RF3] medium speed retriggering, decreasing amplitude (echo effect)
\end{description}

\section{S: Sweep/Shape}

This command has different effects for different instrument types.

\subsection{Pulse Instruments}

S modulates pitch, using the Game Boy hardware. It is useful for creating bass drums and percussion. The first digit affects pitch, the second changes pitch bend velocity.

Note: S has no effect when being used in pulse channel 2!

\subsection{Kit Instruments}

S changes the loop points. The first digit modulates the offset value; the second digit modulates the loop length. (1-7=increase, 9-\$F=decrease.) Used creatively, this command can be very useful for creating a wide range of percussive and timbral effects.

\subsection{Noise Instruments}

Alters noise shape (see section~\ref{noise-instrument-parameters}).
The command is relative, meaning that the digits are independently added to the active noise shape.

\section{T: Tempo}

Change the tick frequency so that the given \textsc{tempo} will be produced (in BPM). The \textsc{tempo} setting will be accurate only if the active groove has 6 ticks per note step. If the groove has some other number of ticks per note step, the \textsc{tempo} value should be adjusted according to the formula
\begin{math}
lsdj\_bpm = (desired\_bpm \times ticks\_per\_step)/{6}
\end{math}.

Setting a value of \$00-\$27 will set the tempo to 256-295 BPM.

\begin{description}
\item Example:
\item[T80] set tempo to 128 BPM
\item[T00] set tempo to 256 BPM
\end{description}

\section{V: Vibrato}

Add vibrato. Not available for noise instruments. The vibrato speed and shape depends on the instrument's \textsc{pitch} setting.

\begin{description}
\item Example:
\item[V42] period=4, depth=2
\item[V00] reset vibrato
\end{description}

\section{W: Wave}

\subsection{For Pulse Instruments:}
Changes waveform.

\subsection{For Wave Instruments:}
The first digit sets synth sound speed, the second sets synth sound length. 0 = no change. The synth will be restarted if length is changed.

\section{Z: RandomiZe}

The Z command repeats the last non-Z command, adding a random number to the original command value. The Z value controls the maximum value of each digit to be added (each digit is added separately).

\begin{description}
\item Example:
\item[Z02] adds one of 0, 1, 2 to the original value.
\item[Z20] adds one of 0, 10, 20 to the original value.
\item[Z22] adds one of 0, 1, 2, 10, 11, 12, 20, 21, 22 to the original value.
\end{description}

Note: Randomize does not work with mayBe, Hop, Groove or Delay commands at the moment.
